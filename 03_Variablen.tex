% Autor: Simon May
% Datum: 2016-10-13

% Der Befehl \newcommand kann auch benutzt werden, um Variablen zu definieren:

% Nummer des Versuchs (z.B. M2):
\newcommand{\varNum}{S1}
% Name des Versuchs:
\newcommand{\varName}{Was ist Experimentieren?}
% Name des Versuchs (kurz, z.B. für Kopfzeile):
\newcommand{\varNameShort}{Was ist Experimentieren?}
% Ist der Versuchstitel sehr lang? (Verringert Schriftgröße des Titels, falls
% „true“)
\newcommand{\varLongTitle}{false}
% Datum der Durchführung:
\newcommand{\varDate}{\DTMdate{2017-10-17}}
% Autoren des Protokolls:
\newcommand{\varAuthor}{Pia Dillmannn, Jonas Lübken}
% Nummer der eigenen Gruppe:
\newcommand{\varGroup}{Gruppe 14}
% E-Mail-Adressen der Autoren (kommagetrennt ohne Leerzeichen!):
\newcommand{\varEmail}{p\_dill02@wwu.de, j\_lueb11@wwu.de}
% E-Mail-Adressen anzeigen (true/false):
\newcommand{\varShowEmail}{true}
% Kopfzeile anzeigen (true/false):
\newcommand{\varShowHeader}{true}
% Inhaltsverzeichnis anzeigen (true/false):
\newcommand{\varShowTOC}{true}
% Literaturverzeichnis anzeigen (true/false):
\newcommand{\varShowBibliography}{true}

